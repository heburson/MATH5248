   \documentclass[12pt]{article}
\usepackage[utf8]{inputenc}
\usepackage{amsmath,amsthm}
\usepackage{fullpage}
\usepackage{amsfonts}
\usepackage{amssymb,multicol}
\usepackage[colorlinks=true,urlcolor=blue]{hyperref}
\usepackage{enumitem}
\usepackage{xcolor}

\newcommand{\Z}{\mathbb{Z}}
\newtheorem*{theorem}{Theorem}

\newlist{checklist}{itemize}{2}
\setlist[checklist]{label=$\square$}

\begin{document}
\begin{center}
{\Large Homework 4}\\
Due: Friday,  February 18 at noon\\


\end{center}
{\bf Instructions:} Submit a pdf of your solutions to the HW 4 assignment on Gradescope. 



\begin{enumerate}
\item[0.] If you would like any of these problems to be graded for proficiency with the core skills, list the skill and the corresponding problem. 
\item Let $p$ be a prime number.  For what integers $x$, is $x^2\equiv 2x\pmod{p}$? Prove that the desired congruence holds for those values of $x$.  (Hint: you should be able to express your answer as a set of two equivalence classes modulo $p$ or a set of two statements $x\equiv ?\pmod{p}$.) In solving this problem, complete the following steps:
\begin{enumerate}
\item State a conjecture about which values of $x$ give you $x^2\equiv 2x\pmod{p}$.
\item Choose two specific values of $p$ and confirm that your conjecture holds for these primes.  Specifically, if you conjecture that $x\equiv a\pmod{p}$ or $x\equiv b\pmod{p}$, confirm that $a^2\equiv 2a\pmod{p}$, $b^2\equiv2b\pmod{p}$, and $x^2\not\equiv 2x\pmod{p}$ if $x\not\equiv a,b\pmod{p}$. 
\item Prove that for any prime $p$, $x^2\equiv 2x\pmod{p}$ if and only if $x$ is in one of your conjectured equivalence classes modulo $p$. 
\end{enumerate}

Note 1: To fully prove that you have the right answer, you would also need to show that if $x^2\equiv 2x\pmod{p}$, then $x$ is in one of  solutions your conjectured equivalence classes. However, this statement is harder to prove and less enlightening, so it is not required for this assignment.

Note 2: This problem will be worth 6 points--2 points for parts (a) and (b) and $4$ points for part (c).
\item Find all solutions to the linear congruence equation $$15x\equiv 6\pmod{9}$$ where $0\le x<9$ by using the following process:
\begin{enumerate}
\item Remark how many solutions you expect to find based on the theorem we learned in class.
\item Find the solutions to the linear congruence equation \begin{equation}\label{lce}
5x\equiv 2\pmod{3}.
\end{equation}
\item Write your solutions as $x=3q+r$ where $q$ is a variable representing an integer and $r$ is a a specific integer determined by your solution to \eqref{lce}.
\item Which values of $q$ result in an $x$ in the desired range?
\item State all of the desired solutions.
\end{enumerate}
\item Prove that, if $\gcd(a,b)=d$, then $\gcd(\frac{a}{d},\frac{b}{d})=1$.  (Hint: Bezout's identity can help you here.)
\item (Exercise 1.7.20 in your textbook) Show that a two-round affine cipher (by two-round, we mean encrypting using an affine cipher and then encrypting the result using another affine cipher) can actually have no net effect in some cases. Specifically, show that, for any integer $x$ with $0\le x\le 25$, $E_{25,25}(E_{25,25}(x))=x$. 
\item Read \href{https://blogs.scientificamerican.com/roots-of-unity/prime-factorization-as-verse/}{this article about poetry based on the Fundamental Theorem of Arithmetic}.  Then, take a stab at writing your own poem using the same FTA structure. For full credit, your poem should satisfy the following specifications/guidelines:
\begin{itemize}
\item Follow a similar structure to Glaz's poem shown in the article (i.e.  Write lines for each prime, pick conjunctions to represent exponentiation and multiplication, then put that together in a poem)
\item Be between 10 and 20 lines (the example poem in the article is 13 lines)
\end{itemize}
Be as creative as you want! Any poem that satisfies the guidelines will demonstrate understanding of the fundamental theorem of arithmetic and, thus, earn full credit. 
\end{enumerate}

\end{document}
