   \documentclass[12pt]{article}
\usepackage[utf8]{inputenc}
\usepackage{amsmath,amsthm}
\usepackage{fullpage}
\usepackage{amsfonts}
\usepackage{amssymb,multicol}
\usepackage[colorlinks=true,urlcolor=blue]{hyperref}
\usepackage{enumitem}
\usepackage{xcolor}

\newcommand{\Z}{\mathbb{Z}}
\newtheorem*{theorem}{Theorem}

\newlist{checklist}{itemize}{2}
\setlist[checklist]{label=$\square$}

\begin{document}
\begin{center}
{\Large Homework 3}\\
Due: Friday,  February 11 at noon\\


\end{center}
{\bf Instructions:} Submit a pdf of your solutions to the HW 3 assignment on Gradescope. 


\noindent {\bf Note:} Problems 4 and 5 require you to find multiplicative inverses. Since you are showing your skills in doing that by hand in problem 2, you can use whatever technique/computer tool you would like to find the needed inverses for problems 4 and 5.  (In SAGE, the code $inverse\_mod(a,m)$ computes the multiplicative inverse of a modulo m.)

\begin{enumerate}
\item[0.] If you would like any of these problems to be graded for proficiency with the core skills, list the skill and the corresponding problem. 
\item Use the Euclidean Algorithm to find $\gcd(65330,5420)$.
\item Use the Euclidean Algorithm to find a multiplicative inverse of $206$ modulo $5427$. 
\item Prove that, if $a_1$ and $a_2$ are units modulo $m$, then $a_1a_2$ is also a unit modulo $m$. 
\item Consider an affine cipher with key $(5,4)$. 
\begin{enumerate}
\item Encrypt the word ``cryptology" using that cipher. 
\item You recieve the ciphertext ``DAROVSWR". What is the decrypted plaintext?
\end{enumerate}

\item Two enemies of yours are passing messages using an affine cipher. You know that they always write formal notes starting with the greating ``Hello" in the plaintext.  You intercept a ciphertext that starts with ``fkhhc." What key did they use?

\item (Exercise 1.6.16 in your textbook) Prove that $x^2-y^2=102$ has no integer solutions. (Hint: Use the contrapositive of the following: If $a,b$ are integers such that $a=b$, then $a\equiv b\pmod{m}$ for all nonzero integers $m$.)
\end{enumerate}

\end{document}