\documentclass[11pt,addpoints,letterpaper]{exam}
\usepackage{amsmath,amssymb}
\usepackage[margin=1in]{geometry}
\usepackage{enumitem,booktabs}
\usepackage{amsfonts,amsthm, systeme,calc}
\usepackage[colorlinks]{hyperref}
\usepackage{tikz}

\setlength{\parindent}{0pt}

\newtheorem*{axiom}{Axiom}

%%%%%%%%%% Exam formatting %%%%%%%%%%%%%%%%%%%%%%%%%%

\newcommand{\answerblank}[2]{
\begin{tikzpicture}
\ifprintanswers
\draw (0,0) -- node[anchor=south, inner sep=1pt] {#2} (#1,0);
\else
\draw (0,0) --  (#1,0);
\fi
\end{tikzpicture}
}

\bracketedpoints
\renewcommand{\questionlabel}{\textbf{\thequestion.}}
\renewcommand{\partlabel}{\textbf{\thepartno.}}
\renewcommand{\subpartlabel}{\textbf{\thesubpart.}}

\qformat{Question \thequestion\dotfill \emph{\totalpoints\ points}}

%\newcommand{\quest}[1][0]{\ifnum{#1}>0 \question[#1] \else \noaddpoints \question[\totalpoints] \addpoints \fi}

\vqword{Problem}
\vtword{Total}
\cellwidth{1in}

\pagestyle{head}

\firstpageheader{}{}{}
\runningheader{\scriptsize\textit{Math 5248 --- Midterm Exam 1 (Spring 2022)}}{}{\scriptsize\textit{page \thepage}}
\runningheadrule

\extrafootheight{-.5in}

%\printanswers

%%%%%%%%%%%%%%%%%%%%%%%%%%%%%%%%%%%%%%%%%%%%%%%%%%%

\begin{document}

\vspace*{-0.5in}

\ifprintanswers
\begin{center}
\fbox{\LARGE\textbf{SOLUTIONS}}
\end{center}
\fi

{\centering

\LARGE Math 5248 --- Midterm Exam 2

}

\smallskip

{\centering

\large Due by 11:59 pm on March 3, 2022

}

\bigskip\bigskip


\hrule

\begin{enumerate}[leftmargin=2em,rightmargin=1em]\setlength{\itemsep}{-2pt}


\item This exam has a total of 15 problems, including 3 challenge problems. All students will be graded on their solutions to Questions 1-12, and students who would like to earn a High Pass must complete one of the challenge problems (13-15). 


\item Your completed exam must be uploaded to Canvas by 11:59 pm central standard time on March 3, 2022.  

\item You may use the Garrett textbook, the online textbook \href{http://math.gordon.edu/ntic/ntic/frontmatter-1.html}{Number Theory: In Context and Interactive}, SAGE or another computation tool (such as wolframAlpha), anything linked/posted in the Lecture Schedule page on Canvas, and help from Prof. Burson.

\item You may not use other sources including, but not limited to, forums such as StackExchange and MathOverflow, cites such as Chegg and Course Hero, or other students. 

\item You must clearly explain your answers at a level that another MATH 5248 student could understand.  For examples of sufficient explanation, you may  want to look at the Homework 1 and Homework 2 solutions posted on Canvas.  

\item You will not get credit for solutions that use major results that we have not covered yet in the course (email Prof. Burson if you are unsure of what counts as a major result). 

\item If you have questions for Prof. Burson, \href{https://calendly.com/hburson/meetings-with-dr-burson}{you may sign up for an appointment here}. 

\item Details about how your work will be assessed can be found on the next page. 

\item If you would like to have some of your problems graded for proficiency please write a list of all problems you would like graded for proficiency credit and which skill(s) you used in each of those problems. \\
\emph{Proficiency credit information:}



\end{enumerate}



\vfill

(If you submit your exam on separate sheets of paper, please copy this statement in full and sign.) 
\vspace{2em}\\
\emph{I neither gave nor received aide aside from the resources specifically allowed in item 3 during this exam.}

\vspace{1cm}

\hfill\textbf{Signature:} \answerblank{3in}{}

\vfill

\newpage
\begin{center}
\large
How the exam will be assessed
\end{center}
Each part of questions 1-11 will be graded using one of the rubrics below. In the instructions for question 12, there is a summary of how your work will be graded.
\begin{center}
\begin{tabular}{|p{3.5cm}|p{2.7cm}|p{2.7cm}|p{2.5cm}|p{2.8cm}|}
\hline \multicolumn{5}{|c|}{4-point rubric}\\ \hline
4 & 3 & 2 &1 & 0\\ \hline
Correct and clearly communicated. Solution/proof demonstrates clear understanding of the concepts. If the problem requires a proof, the proof clearly states all assumptions, defines all variables, and notes where definitions or theorems are used. & Solution/proof demonstrates understanding of the concepts, but might have one minor issue such as a small computational error or a small jump in logic.  & Solution/proof demonstrates partial understanding, but has a large jump in logic or serious writing flaws such as incomplete sentences in a proof.  & Solution/proof shows effort but it has serious flaws in mathematical logic.  &Solution is not submitted, is illegible, or does not demonstrate evidence of understanding. \\ \hline
\end{tabular}
\begin{tabular}{|p{3.5cm}|p{2.7cm}|p{2.8cm}|p{2.5cm}|}
\hline \multicolumn{4}{|c|}{3-point rubric}\\ \hline
 3 & 2 &1 & 0\\ \hline
Solution demonstrates clear understanding of the concepts. Work and exposition allow the reader to easily follow the logic. & Solution demonstrates partial, but significant understanding of the concepts. It might have one or two minore issues such as computational errors or a small piece of missing justification. & Solution shows some understanding but it has serious flaws in mathematical logic.  &Solution is not submitted, is illegible, or does not demonstrate evidence of understanding. \\ \hline
\end{tabular}
\end{center}

As stated in the syllabus, you will earn a High Pass, a Pass, a No Pass, or an Incomplete on this exam. Here is how to earn each grade:
\\[1em]
High pass: You can earn a high pass by earning at least 53 of the 58 points on this exam and correctly completing one of the three challenge problems. \\[5pt]
Pass: You can earn a pass on this exam by earning at least 40 of the 58 points on this exam. \\[5pt]
No pass: If you obtain between 15 and 40 points, you will earn a No Pass.\\[5pt]
Incomplete: If you do not meet the criteria for a No Pass, you will earn an incomplete. 
\\[3em]
{\bf Revision Policy} Once you get your exam back with my feedback, you will be able to revise your solution for up to 3 questions. 
\vspace{\stretch{1}}

\gradetablestretch{1.35}

{\centering

\partialpointtable{1} 
\\[2em]
You may find the following useful:
\begin{align*}
&A\hspace{.35cm} B\hspace{.35cm}C\hspace{.35cm}D\hspace{.35cm}E\hspace{.35cm}F\hspace{.35cm}G\hspace{.35cm}H\hspace{.35cm}I\hspace{.35cm}J\hspace{.35cm}K\hspace{.35cm}L\hspace{.35cm}M\hspace{.35cm}N\hspace{.35cm}O\hspace{.35cm}P\hspace{.35cm}Q\hspace{.35cm}R\hspace{.35cm}S\hspace{.35cm}T\hspace{.35cm}U\hspace{.35cm}V\hspace{.35cm}W\hspace{.35cm}X\hspace{.35cm}Y\hspace{.35cm}Z\\
&0\hspace{.46cm} 1\hspace{.46cm} 2\hspace{.46cm} 3\hspace{.46cm} 4\hspace{.46cm} 5\hspace{.46cm} 6\hspace{.46cm} 7\hspace{.46cm} 8\hspace{.46cm} 9\hspace{.35cm} 10\hspace{.285cm} 11\hspace{.285cm} 12\hspace{.285cm} 13\hspace{.285cm} 14\hspace{.285cm} 15\hspace{.285cm}16\hspace{.285cm}17\hspace{.285cm}18\hspace{.285cm}19\hspace{.285cm}20\hspace{.285cm}21\hspace{.285cm}22\hspace{.285cm}23\hspace{.285cm}24\hspace{.285cm}25
\end{align*}
}

\vfill





\newpage

\begin{questions}
\begingradingrange{1}
\question These questions are computational. You should include enough work and justification to show that you understand the properties of multiplication modulo $m$ and the differences between the notation ``$\%m$" and ``$\!\!\!\mod m$". Note that, it is possible to do this problem without using $\mod m$ notation, but you are allowed to use equivalences modulo $m$  if they help you explain your process. 
\begin{parts}
\part[3] Compute  $(53\cdot 705)\%7$ without computing $53\cdot 705$. 
\vfill
\part[3] Compute $(53^5)\%55$ without computing $53^5$.
\vfill
\end{parts}
\newpage
\question[4] Let $N,m$ be integers with $m\ne 0$. Prove that 
$N\%(-m)=N\%m.$
\vfill
\question[4] Prove or disprove the following statement:\\[1em]
If $a,b,m$ are integers with $m>0$, $a\equiv b\pmod m$ and $n$ is an integer such that $0<n<m$, then $a\equiv b\pmod n$. 
\vfill
\question[4] Prove or disprove the following statement:\\[1em]
 If $a\equiv b\pmod{m}$ and $d|m$, then $a\equiv b\pmod d$. 
 \vfill 
 \newpage
 \question[4] Prove that $x^2+3y^7+6z^6 = 2$ has no solutions for integers $x,y,z$. 
 \vfill

\question[4] Prove the following statement: \\[1em]
If $p$ is prime and $x$ is an integer such that $x^2\equiv 3x\pmod{p}$, then $x\equiv 0\pmod{p}$ or $x\equiv 3\pmod{p}$. 
\vfill
    \newpage



   
    \question
    \begin{parts} \part[3] Use the Euclidean Algorithm to determine the greatest common divisor of $2022$ and $226$.  

         
       
          
          \vfill
          
          \part[3] Are there integers $x$ and $y$ satisfying the equation $15= 2022x+226y$?  If so, find such an $x$ and $y$.  If not, explain how you know that such an $x$ and $y$ cannot exist.  
    \vfill      
       
          \end{parts}
          
          \bigskip
   
  
   \newpage
    \question

  \begin{parts}
\part[3]  Use the Euclidean Algorithm to find an inverse to $55$ modulo $71$. 
\vfill
  
\part[3]   Let $n$ be a nonzero integer. Find an inverse to $n$ modulo $3n^2+1$.  Explain why your answer is true.  (This explanation could be as brief as one sentence with an equation verifying that your answer satisfies the definition of an inverse.)
   
   \vfill
   
 \end{parts}
  
   
   \newpage
  \question[4] You believe that the plaintext ``SNOW" has been encrypted using an affine cipher to the ciphertext ``YZES". What is the key of the cipher?
 \vfill
 \question[3] Consider an alphabet with four characters: $\spadesuit$, $\heartsuit$, $\diamondsuit$, and $\clubsuit$.  Suppose that you have a stream of characters ${s}$ written in this alphabet which consists of $45$ $\spadesuit$s,  $19$ $\heartsuit$s, $12$ $\diamondsuit$s, and $4$ $\clubsuit$s.  What is $\textrm{I}_{avg}({s})$?
 \vfill
 \newpage
\question[3] You are performing a Friedman attack on a Vigenere cipher. If $y=(y_0,\ldots, y_{N-1})$ is a string of ciphertext of length $N$, let $y^{(+\ell)}=(y_\ell,\cdots, n_{N-1},y_0,\ldots,y_{\ell-1})$ for each $0<\ell<N$ (the $y_0,\ldots,y_{\ell-1}$ at the end allow us to use the full string of text). For each $4<\ell<17$, you computed $I(y,y^{(+\ell)})$ as follows:
$$
\begin{array}{l|l|l|l}
\ell & I(y,y^{(+\ell)}) & \ell & I(y,y^{(+\ell)})\\ \hline
4 & 0.041&10 & 0.061\\
5 & 0.062&11 & 0.039\\
6 & 0.034&12 & 0.038\\
7 & 0.025& 13 & 0.032\\
8 & 0.039&14 & 0.038\\
9 & 0.041 &15 & 0.065
\end{array}
$$
What key length does the Friedman attack predict is most likely based on the data above? Explain your choice.
% \newpage
% \quest[5]  You are trying to attack a Vigenere cipher.  You have already determined that this Vigenere cipher has key length $3$.  If $y = (y_0, \ldots, y_{N-1})$ is a string of ciphertext of length $N$ with $N \equiv 0 \pmod3$, let $y^{(j)} = (y_j,y_{3+j},\ldots,y_{N-j-1})$ be the substring of $y$ whose typical entry is $y_i$ for some $i \equiv j \pmod 3$.  Let $E_t(y^{(j)})$ be the encryption of the string $y^{(j)}$ by the shift cipher with key $t$ for $0 \leq t \leq 25$ and $0 \leq j \leq 2$ (i.e. the string obtained by taking $y^{(j)}$ and shifting each character forward by $t$).  With $MI$ indicating the mutual index of coincidence, you compute the following table:
%
%\begin{center}
%    \begin{tabular}{ l | l | l | l}
%    $t$ & $MI(E_t(y^{(0)}),y^{{(1)}})$ & $MI(E_t(y^{(0)}),y^{{(2)}})$  & $MI(E_t(y^{(1)}),y^{{(2)}})$   \\ \hline
%    5 & 0.067 & 0.031 & 0.036\\ 
%    9 & 0.035 & 0.072 & 0.041\\ 
%    11 & 0.068 & 0.028 & 0.075\\ 
%    16 & 0.036 & 0.069 & 0.039\\ 
%    24 & 0.029 & 0.040 & 0.069\\ 
%    \end{tabular}
%    \end{center} 
%    
%Assume that values of $t$ not appearing in the table above gave rise to very low values of $I(E_t(y^{(j)}),y^{(k)})$ for all $0 \leq j<k \leq 2$.  Given that the key has the form $k = (k_0, k_1, k_2)$, based on the data above, what are all likely values of $k_1-k_0$ and $k_2-k_0$? 
\newpage

\question[10] Write 5 true/false questions that illustrate a variety of ideas from this course that you might put on this exam if you were teaching the class. Give a key, explain the answers, then explain why you chose these particular questions and what you hope they will assess. You can earn 2 points per question based on accuracy of your answer key and the clarity of your explainations.

\vfill
\endgradingrange{1}
\newpage 

\qformat{Challenge Question {\thequestion}\hfill}
{Challenge Problems:} To earn a ``High Pass" on this exam, you must correctly complete {\bf one} of the following problems. This problem will be graded Pass/Fail based on mathematical correctness and clear communication. If you submit a solution to more than one problem, the first one will be graded.  If you only care about earning a ``Pass," you may skip this section.
\begingradingrange{2}
\question You believe that the plaintext ``SNOW" has been encrypted to the ciphertext ``YZMS".  Explain why this encryption could not have been the result of applying an affine cipher. 

 \question 
 \begin{parts} 
 \part Prove or disprove the following statement:\\[1em]
 If $a$ and $b$ are coprime ($\gcd(a,b)=1$) integers and $a|bc$ for some integer $c$, then $a|c$. 
 \part Prove or disprove the following statement:\\[1em]
 If $a$, $b$, and $c$ are integers such that $a \nmid b$ and $a|bc$, then $a|c$. 
\end{parts}
\question  You are trying to attack a Vigenere cipher.  You have already determined that this Vigenere cipher has key length $4$.  If $y = (y_0, \ldots, y_{N})$ is a string of ciphertext of length $N$ with $N \equiv 0 \pmod4$, let $y^{(j)} = (y_j,y_{4+j},\ldots,y_{N-4+j-1})$ be the substring of $y$ whose typical entry is $y_i$ for some $i \equiv j \pmod 4$.  Let $E_t(y^{(j)})$ be the encryption of the string $y^{(j)}$ by the shift cipher with key $t$ for $0 \leq t \leq 25$ and $0 \leq j \leq 2$ (i.e. the string obtained by taking $y^{(j)}$ and shifting each character forward by $t$).  You compute the following table:

\begin{center}
    \begin{tabular}{ l | l | l | l | l }
    $t$ & $I(E_t(y^{(0)}),y^{{(1)}})$ & $I(E_t(y^{(0)}),y^{{(2)}})$  & $I(E_t(y^{(0)}),y^{(3)})$ & $I(E_t(y^{(1)}),y^{(2)})$  \\ \hline
    3 & 0.057 & 0.031 & 0.036 & 0.033\\     
    5 & 0.035 & 0.041 & 0.054 & 0.039\\ 
    8& 0.052 & 0.072 & 0.042 & 0.051\\ 
    11 & 0.064 & 0.058 & 0.039 & 0.037\\ 
    16 & 0.029 & 0.065 & 0.068 & 0.045 \\ 
    23 & 0.035 & 0.037 & 0.051 & 0.061
    \end{tabular}
    \end{center} 
    
Assume that,  values of $t$ not appearing in the table above gave rise values of $I(E_t(y^{(j)}),y^{(k)})$ that are unremarkable for all $0 \leq j<k \leq 2$.  Given that the key has the form $k = (k_0, k_1, k_2,k_3)$, based on the data above, what are all likely values of $k_1-k_0$, $k_2-k_0$, and $k_3-k_0$? For what value(s) of $t$, would you expect $I(E_t(y^{(1)}),y^{(3)})$ to be large? What about $I(E_t(y^{(2)}),y^{(3)})$?
\endgradingrange{2}
\end{questions}


\end{document}