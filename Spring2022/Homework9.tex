 \documentclass[12pt]{article}
\usepackage[utf8]{inputenc}
\usepackage{amsmath,amsthm}
\usepackage{fullpage}
\usepackage{amsfonts}
\usepackage{amssymb,multicol}
\usepackage[colorlinks=true,urlcolor=blue]{hyperref}
\usepackage{enumitem,systeme}
\usepackage{xcolor}

\newcommand{\Z}{\mathbb{Z}}
\newtheorem*{theorem}{Theorem}
\newcommand{\ds}{\displaystyle}

\newlist{checklist}{itemize}{2}
\setlist[checklist]{label=$\square$}

\begin{document}
\begin{center}
{\Large Homework 9}\\
Due: Friday,  April 29 at noon\\


\end{center}
{\bf Instructions:} Submit a pdf of your solutions to the HW 9 assignment on Gradescope. 



\begin{enumerate}
\item[0.] If you would like any of these problems to be graded for proficiency with the core skills, list the skill and the corresponding problem. 


\item Use Euler's criterion to determine whether or not $19$ is a square modulo $107$. 

\item Consider $N=17653=127\cdot 139$. You want to prove that you know the number $v=156$. The verifier knows that $v^2\%N=6683$, but does not know $v$ or the factorization of $N$. You want to prove that you know $v$.  The verifier asks for your sequence of random integers to have length $4$ and tells you that $S=\{1,3,4\}$. If you were using the Fiege-Fiat-Shamir scheme as described in class, what information will you send to show that you know $v$? (Hint: there are infinitely many right answers. Each right answer will consist of two sequences of $4$ integers.)

\item Now, you are playing the role of verifier.  You are given $N=25573$ and that $s=v^2\%N=19083$. The prover sends you the sequence $(17787,20178,108,12769,16129)$. You choose $S=\{1,3\}$ and they reply with the sequence $(17955,843,4185,113,127)$. Did the prover convince you that they know the value of $v$? Why or why not?

%\item Let $p\ge 5$ be a prime. We say that $a$ is a cube modulo $p$ if there is an integer $c$ satisfying $a\equiv c^3\pmod{p}$. 
% \begin{enumerate}
% \item Let $a$ and $b$ be cubes modulo $p$. Prove that $ab$ is a cube modulo $p$. 
% \item Give an example to show that (unlike the case with squares), it is possible for $p\nmid a$, $p\nmid b$ and none of $a$, $b$, and $ab$ to be a cube modulo $p$. 
 %\end{enumerate}
 
 \vspace{2em}
{ The following problems will be graded for proficiency credit only. . You may do as many or as few of them as you would like.}
\item[A.]  Suppose that $k$ is a discrete logarithm base $g$ of $x$ modulo $n$ and that $\ell$ is a discrete logarithm base $g$ of $y$.  Prove that, if $z$ is a discrete logarithm base $g$ of $xy$ modulo $n$, then $z\equiv k+\ell\pmod{\phi(n)}$.  
\item[B.] Fix a prime number $p$ and an integer $n\not\equiv 0,1\pmod{p}$. Set $x=n^{p-2}+n^{p-3}+\cdots+n+1$. Show that $p\mid x$. (Hint: consider expanding $(n-1)x$.)
\item[C.] You believe that the plaintext `stop' has been encoded by a $2\times 2$ Hill cipher to the ciphertext $NCRQ$. What are all possibilities for the key that was used to perform the encryption? If no key is possible, explain why. 
\end{enumerate}

\end{document}