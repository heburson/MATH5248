   \documentclass[12pt]{article}
\usepackage[utf8]{inputenc}
\usepackage{amsmath,amsthm}
\usepackage{fullpage}
\usepackage{amsfonts}
\usepackage{amssymb,multicol}
\usepackage{hyperref, enumitem}
\usepackage{xcolor,graphicx}

\hypersetup{
    colorlinks=true}

\newcommand{\Z}{\mathbb{Z}}
\newtheorem*{theorem}{Theorem}

\newlist{checklist}{itemize}{2}
\setlist[checklist]{label=$\square$}

\begin{document}
\begin{center}
{\Large Homework 2}\\



\end{center}
{\bf Instructions:} Submit a pdf of your solutions to the HW 2 assignment on Gradescope by 11:59pm on Friday February 13. 

Problem 5 is marked as a peer review problem. \href{https://docs.google.com/document/d/1jSw9pmMJJFUx_6dTkcUi4k4HJNIrrIIgs9zdWhAycx0/edit?usp=drive_link}{This document} gives directions and deadlines for the peer review process.

When working on this assignment, you should focus on the following goals:
\begin{itemize}
\item Clearly communicate computational solutions using enough explanation that another 5248 student could follow your work.
\item Perform a known-plaintext attack on the affine cipher. 
\item Use the Euclidean algorithm to find an inverse of one number modulo some integer. 
\item Use examples to determine if a conjecture is true and then prove or disprove the conjecture. 
\item Use the definition of a unit in a proof. 
\item Use the definition of divisibility in a proof. 
\item Write clear and correct proofs that meet the following conventions of a mathematical proof:
\begin{checklist}
\item Written in complete sentences.
\item All assumptions stated at the beginning of the proof.
\item Define variables before/when you use them.
\item Use symbols when writing a precise mathematical formula or equation and English words when not writing a formula (i.e. do not use a symbol, such as $\forall,\, \exists,\, \therefore, \, \implies$ in place of a word or phrase.)
\item Show computational/algebraic work vertically and centered in the page. 
\end{checklist}
\end{itemize}

\begin{enumerate}
\item[0.] If you would like any of these problems to be graded for proficiency with the core skills, list the skill and the corresponding problem. 
\item Consider an affine cipher with key $(5,4)$. 
\begin{enumerate}
\item Encrypt the word ``cryptology" using that cipher. 
\item Decrypt the ciphertext ``NS" using this cipher. Clearly state the linear congruences you are solving and show all of your steps for solving those congruences. 
\end{enumerate}
\item Two enemies of yours are passing messages using an affine cipher. You know that they always write formal notes starting with the greeting ``Hello" in the plaintext.  You intercept a ciphertext that starts with ``fkhhc." What key did they use?

\item Let $n$ be a positive integer. Without using the Euclidean algorithm, prove that $n$ is a unit modulo $4n-1$.   


\item Use the Euclidean Algorithm to find a multiplicative inverse of $206$ modulo $5427$. 

\item (Peer review problem) Suppose that $a=bq+r$, for some integers $a,b, q$ and $r$. Without using any properties of the $\gcd$ besides the definition as the largest common divisor of $x$ and $y$, prove that $\gcd(a,b)=\gcd(b,r)$.  (Hint: Prove that the set of common divisors of $a$ and $b$ is exactly the same as the set of common divisors of $b$ and $r$.)



\end{enumerate}

\end{document}
