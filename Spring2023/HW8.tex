 \documentclass[12pt]{article}
\usepackage[utf8]{inputenc}
\usepackage{amsmath,amsthm}
\usepackage{fullpage}
\usepackage{amsfonts}
\usepackage{amssymb,multicol}
\usepackage[colorlinks=true,urlcolor=blue]{hyperref}
\usepackage{enumitem}
\usepackage{xcolor,systeme}

\newcommand{\Z}{\mathbb{Z}}
\newcommand{\ds}{\displaystyle}
\newtheorem*{theorem}{Theorem}

\newlist{checklist}{itemize}{2}
\setlist[checklist]{label=$\square$}

\begin{document}
\begin{center}
{\Large Homework 8}\\
Due: Friday,  April 7 at 11:59pm\\


\end{center}
{\bf Instructions:} Submit a pdf of your solutions to the HW 8 assignment on Gradescope.  



\begin{enumerate}
\item[0.] If you would like any of these problems to be graded for proficiency with the core skills, list the skill and the corresponding problem. 

\item Let $p$ and $q$ be distinct primes.  Show that for all $x \in \mathbb{Z}$, we have the congruence $x^{(p-1)(q-1)+1} \equiv x \pmod{pq}$.  (Hint: Use the Chinese Remainder Theorem to reframe the desired congruence as a system of two congruences--one modulo $p$ and one modulo $q$.)

\item Determine the order of $12$ modulo $35$ without computing $12^a\%35$ for more than $7$ values of $a$

\item Prove that, if $a$ is a primitive root modulo $p$, then $a^{-1}$ (the multiplicative inverse of $a$ modulo $p$) is also a primitive root modulo $p$. 

\item Explain why $5$ is not a primitive root modulo $20$ without computing \emph{any} of the powers of $5$ modulo $20$. 

\item Determine whether or not $3$ is a primitive root modulo $19$ without computing all of the powers of $3$ modulo $19$. 





\end{enumerate}


\end{document}