 \documentclass[12pt]{article}
\usepackage[utf8]{inputenc}
\usepackage{amsmath,amsthm}
\usepackage{fullpage}
\usepackage{amsfonts}
\usepackage{amssymb,multicol}
\usepackage[colorlinks=true,urlcolor=blue]{hyperref}
\usepackage{enumitem}
\usepackage{xcolor,systeme}

\newcommand{\Z}{\mathbb{Z}}
\newcommand{\ds}{\displaystyle}
\newtheorem*{theorem}{Theorem}

\newlist{checklist}{itemize}{2}
\setlist[checklist]{label=$\square$}

\begin{document}
\begin{center}
{\Large Homework 9}\\
Due: Friday,  April 14 at 11:59pm\\


\end{center}
{\bf Instructions:} Submit a pdf of your solutions to the HW 9 assignment on Gradescope.  



\begin{enumerate}
\item[0.] If you would like any of these problems to be graded for proficiency with the core skills, list the skill and the corresponding problem. 

\item Bob and Alice would like to use the Diffie-Hellman Key Exchange to agree on a session key $k$. Alice and Bob agree on the prime $6829$ and that the base $2$ modulo $6829$. Bob chooses a secret random number $x$ and tells Alice that $2^x\%6829$ is $5792$. Alice chooses $3$ as her secret exponent.  Answer the following questions without computing the value of $x$. 
\begin{enumerate}
\item What is the shared secret key $k$ that Alice and Bob will be using?
\item You are able to gain access to the network Alice and Bob are using to communicate, so you decide to implement a interceptor attack with exponent $c=10$. What key does Bob think he and Alice agreed to? What key does Alice think they agreed to?
\end{enumerate}	

\item You are setting up an RSA cipher with modulus $5021131$.  You may use the fact that $5021131 = 1907(2633)$.  (You may want to use a computer to do most of the computation on this problem.)
\begin{enumerate}
\item Is $5$ a valid encryption key?  If so, find the corresponding decryption key.  If not, explain why not.

\item Is $7$ a valid encryption key?  If so, find the corresponding decryption key.  If not, explain why not.

\item The encryption key $e = 3$ is valid for the modulus $5021131$ and corresponds to the decryption key $d = 3344395$.  If you are setting up a public key with this information, what is all of the information that you would post publicly so that a friend could send you an encrypted message?  What is all of the information relevant to this cipher that you would keep secret?


\item If your plaintext is represented by $x = 884204$, use the encryption key $e = 3$ to encrypt $x$.

\item You have received the ciphertext message $y = 384$.  Use the decryption key $d = 3344395$ to decrypt the message.  
\end{enumerate}

\item Use Euler's criterion to determine whether or not $19$ is a square modulo $107$. 

\item You know that an enemy has been encrypting messages using RSA and that the recipient's RSA modulus is $n = 580,799$.  (Because the RSA modulus is public information, this setup is plausible except for the very small size of $580,799$ as an RSA modulus.)  You would like to perform an attack on their RSA setup so that you can read their encrypted messages.  Magically, you have access to a square root oracle!  (This part of the setup is not plausible.  If square root oracles existed, RSA would not be secure.)  
\begin{enumerate}
\item You compute $1258^2 \equiv 420,966 \pmod{580,799}$ and ask your magical square root oracle for another square root of $420,966$ modulo $580,799$.  Your square root oracle tells you that $579541^2 \equiv 420,966 \pmod{580,799}$  Can you use the information given to you by the square root oracle to factor $580,799$?  If so, do it.  If not, explain why not.

\item You compute $1234^2\equiv 361,158 \pmod{580,799}$ and ask your magical square root oracle for another square root of $361,158 $ modulo $580,799$. Your square root oracle tells you that $83,340^2\equiv 361,158 \pmod{580,799}$.  Can you use the information given to you by the square root oracle to factor $580,799$?  If so, do it.  If not, explain why not.
\end{enumerate}
\end{enumerate}


\end{document}