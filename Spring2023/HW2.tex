   \documentclass[12pt]{article}
\usepackage[utf8]{inputenc}
\usepackage{amsmath,amsthm}
\usepackage{fullpage}
\usepackage{amsfonts}
\usepackage{amssymb,multicol}
\usepackage[colorlinks=true,urlcolor=blue]{hyperref}
\usepackage{enumitem}
\usepackage{xcolor}

\newcommand{\Z}{\mathbb{Z}}
\newtheorem*{theorem}{Theorem}

\newlist{checklist}{itemize}{2}
\setlist[checklist]{label=$\square$}

\begin{document}
\begin{center}
{\Large Homework 2}\\
Due: Friday,  February 10 at 11:59 ,pm\\


\end{center}
{\bf Instructions:} Submit a pdf of your solutions to the HW 2 assignment on Gradescope. 

When working on this assignment, you should focus on the following goals:
\begin{itemize}
\item Demonstrate that you understand how to write a complete set of equivalence classes modulo $m$ and when two equivalence classes are equal. 
\item Demonstrate understanding of the definition of a unit modulo $m$.  
\item Use examples to determine if a conjecture is true and then prove or disprove the conjecture. 
\item Write clear and correct proofs that meet the Writing Guidelines posted on Canvas.
\end{itemize}

\begin{enumerate}
\item[0.] If you would like any of these problems to be graded for proficiency with the core skills, list the skill and the corresponding problem. 
\item Write out all elements of $\Z_{8}$ using two different collections of representatives for the equivalence classes.  (Note: Your answer will have two different sets; each equivalence class in the first set should be equal to an equivalence class in the second set.)
\item Make a multiplication table for $\Z_{10}$. Use it to identify the units (invertible elements) of $\Z_{10}$.  What is $\phi(10)$?
\item Let $n$ be a positive integer. Prove that $n$ is a unit modulo $5n-1$. 

\item Prove or disprove each of the following statements.
\begin{enumerate}
\item For positive integers $n$ and $N$ and any integer $x$, $$(x\% N)\% n= x\% n.$$
\item For integers $a$, $b$, $c$, and $m$ with $m,c>0$, if $a\equiv b\pmod{mc}$, then $a\equiv b\pmod{m}$. 
\end{enumerate} 
\item Suppose that $a=bq+r$, for some integers $a,b, q$ and $r$. Without using any properties of the $\gcd$ besides the definition as the largest common divisor of $x$ and $y$, prove that $\gcd(a,b)=\gcd(b,r)$.  (Hint: $x=y$ if and only if $x\le y$ and $x\ge y$.)

\end{enumerate}

\end{document}