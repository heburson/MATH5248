 \documentclass[12pt]{article}
\usepackage[utf8]{inputenc}
\usepackage{amsmath,amsthm}
\usepackage{fullpage}
\usepackage{amsfonts}
\usepackage{amssymb,multicol}
\usepackage[colorlinks=true,urlcolor=blue]{hyperref}
\usepackage{enumitem}
\usepackage{xcolor,systeme}

\newcommand{\Z}{\mathbb{Z}}
\newcommand{\ds}{\displaystyle}
\newtheorem*{theorem}{Theorem}

\newlist{checklist}{itemize}{2}
\setlist[checklist]{label=$\square$}

\begin{document}
\begin{center}
{\Large Homework 7}\\
Due: Friday,  March 31 at 11:59pm\\


\end{center}
{\bf Instructions:} Submit a pdf of your solutions to the HW 7 assignment on Gradescope.  



\begin{enumerate}
\item[0.] If you would like any of these problems to be graded for proficiency with the core skills, list the skill and the corresponding problem. 


\item  You are performing a ciphertext-only attack on a ciphertext that you know was encrypted using a Vigenere cipher.  If $y = (y_1, \ldots, y_{N})$ is a string of ciphertext of length $N$, let $$y_{\ell} = (y_1, y_{1+\ell}, y_{1+2\ell}, \ldots, y_{1+\ell\lfloor{\frac{N-1}{\ell}}\rfloor})$$ for each $0 < \ell <N$ ($\lfloor x \rfloor$ is the largest integer $\le x$).  Consider the following table of values of $\ell$ and $\mathrm{IndCo}(y_{\ell})$.

\begin{center}
    \begin{tabular}{ l | l | l | l }
    $\ell$ & $\mathrm{IndCo}(y_\ell)$ &  $\ell$ & $\mathrm{IndCo}(y_\ell)$  \\ 
    \hline
    3 & 0.035 & 9  & 0.048\\ 
    4 & 0.039 & 10 & 0.068  \\ 
    5 & 0.063 & 11 & 0.034  \\
    6 & 0.041  & 12 & 0.031  \\
    7 & 0.022  &  13 & 0.050\\
    8 & 0.033  &  14 & 0.039\\
    \end{tabular}
\end{center}

Based on the table above, what do you think the key length is?  Explain your answer. (You may assume that your ciphertext is significantly longer than the key length.)

\item   You are trying to attack a Vigenere cipher.  You have already determined that this Vigenere cipher has key length $4$.  If $\underline{s} = (s_1, \ldots, s_{N})$ is a string of ciphertext of length $N$ with $N \equiv 0 \pmod4$, let $s^{(j)} = (s_j,s_{4+j},s_{8+j},\ldots,s_{N-4+j})$ be the substring of $s$ whose typical entry is $s_i$ for some $i \equiv j \pmod 4$.  Let $E_t(s^{(j)})$ be the encryption of the string $s^{(j)}$ by the shift cipher with key $t$ for $0 \leq t \leq 25$ and $1 \leq j \leq 4$ (i.e. the string obtained by taking $s^{(j)}$ and shifting each character forward by $t$ letters).  You compute the following table:

\begin{center}
    \begin{tabular}{ l | l | l | l }
    $t$ & $\mathrm{MutIndCo}(s^{(1)},E_t(s^{{(2)}}))$ & $\mathrm{MutIndCo}(s^{(1)},E_t(s^{{(3)}}))$  & $\mathrm{MutIndCo}(s^{(1)},E_t(s^{{(4)}}))$  \\ \hline
    3 & 0.057 & 0.031 & 0.036 \\     
    5 & 0.035 & 0.041 & 0.054 \\ 
    8& 0.052 & 0.072 & 0.042\\ 
    11 & 0.064 & 0.058 & 0.039\\ 
    16 & 0.029 & 0.065 & 0.068 \\ 
    23 & 0.035 & 0.037 & 0.051
    \end{tabular}
      \begin{tabular}{ l | l | l}
    $t$ & $\mathrm{MutIndCo}(s^{(2)},E_t(s^{{(3)}}))$ & $\mathrm{MutIndCo}(s^{(2)},E_t(s^{{(4)}}))$\\ \hline
    3 & 0.033 & 0.041 \\     
    5 & 0.039 & 0.069\\ 
    8 & 0.051 & 0. 038\\ 
    11 & 0.037 & 0.061\\ 
    16 & 0.045 & 0.052\\ 
    23 & 0.061 & 0.035\\
    \end{tabular}
    \end{center} 
    
For the questions below, assume that,  values of $t$ not appearing in the table above gave rise values of $\mathrm{MutIndCo}(s^{(j)},E_t(s^{(k)}))$ that are unremarkable for all $1 \leq j<k \leq 4$.  
\begin{enumerate}
\item Given that the key has the form $k = (k_0, k_1, k_2,k_3)$, based on the data above, what are all likely values of $k_1-k_0$, $k_2-k_0$, and $k_3-k_0$? Explain your answer.
\item  For what value(s) of $t$, would you expect $\mathrm{MutIndCo}(s^{(2)},E_t(s^{(3)}))$ to be large? Explain your answer.
\end{enumerate}

\item Prove that the equation $r^{16}+17t=11$ has no solutions for integers $r$ and $t$.  (Hint: Fermat's Little Theorem is useful here.)

\item In this problem you will prove that the two statements of Fermat's Little Theorem are equivalent. See the parts below for precise statements of what you must prove. 
\begin{enumerate}
\item Prove that,  if \(p\) is prime and $$a^{p-1}\equiv \begin{cases} 0 \pmod{p} & \text{if } p\mid a \\ 1 \pmod{p} & \text{if } p\nmid a \end{cases},$$ then $$a^p\equiv a\pmod{p}.$$
\item Prove that, if \(p\) is prime and $$a^p\equiv a\pmod{p},$$ then $$a^{p-1}\equiv \begin{cases} 0 \pmod{p} & \text{if } p\mid a \\ 1 \pmod{p} & \text{if } p\nmid a \end{cases}.$$
\end{enumerate}

\end{enumerate}


\end{document}
