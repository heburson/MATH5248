 \documentclass[12pt]{article}
\usepackage[utf8]{inputenc}
\usepackage{amsmath,amsthm}
\usepackage{fullpage}
\usepackage{amsfonts}
\usepackage{amssymb,multicol}
\usepackage[colorlinks=true,urlcolor=blue]{hyperref}
\usepackage{enumitem}
\usepackage{xcolor,systeme}

\newcommand{\Z}{\mathbb{Z}}
\newcommand{\ds}{\displaystyle}
\newtheorem*{theorem}{Theorem}

\newlist{checklist}{itemize}{2}
\setlist[checklist]{label=$\square$}

\begin{document}
\begin{center}
{\Large Homework 5}\\
Due: Friday,  March 17 at 11:59pm\\


\end{center}
{\bf Instructions:} Submit a pdf of your solutions to the HW 5 assignment on Gradescope.  



\begin{enumerate}
\item[0.] If you would like any of these problems to be graded for proficiency with the core skills, list the skill and the corresponding problem. 

\item For the following values of $y$ and $p$, determine whether or not $y$ is a square modulo $p$. If it is a square, find the principal square root of $y$ modulo $p$.  If it is not a square, justify your reasoning with a proof.
\begin{enumerate}
\item $y=7, \ p=19$
\item  $y=7, \ p=23$
\end{enumerate}

 \item  A friend sends you the encrypted message $XI$ using a Hill cipher whose key you and your friend agreed at a previous meeting would be $K = \begin{pmatrix}
1 & 2\\ 
4 & 17
\end{pmatrix}$.  What is the plaintext of the message your friend sent you?\\


 \item Read \href{https://blogs.scientificamerican.com/roots-of-unity/prime-factorization-as-verse/}{this article about poetry based on the Fundamental Theorem of Arithmetic}.  Then, write your own poem using the same FTA structure. For full credit, your poem should satisfy the following specifications/guidelines:
\begin{itemize}
\item Follow a similar structure to Glaz's poem shown in the article (i.e.  Write lines for each prime, pick conjunctions to represent exponentiation and multiplication, then put that together in a poem)
\item Be between 10 and 20 lines (the example poem in the article is 13 lines)
\end{itemize}
Be as creative as you want! Any poem that satisfies the guidelines will demonstrate understanding of the fundamental theorem of arithmetic and, thus, earn full credit. 

\end{enumerate}

\end{document}