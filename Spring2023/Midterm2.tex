\documentclass[11pt,addpoints,letterpaper]{exam}
\usepackage{amsmath,amssymb}
\usepackage[margin=1in]{geometry}
\usepackage{enumitem,booktabs}
\usepackage{amsfonts,amsthm, systeme,calc}
\usepackage[colorlinks]{hyperref}
\usepackage{tikz}

\setlength{\parindent}{0pt}

\newtheorem*{axiom}{Axiom}

%%%%%%%%%% Exam formatting %%%%%%%%%%%%%%%%%%%%%%%%%%

\newcommand{\answerblank}[2]{
\begin{tikzpicture}
\ifprintanswers
\draw (0,0) -- node[anchor=south, inner sep=1pt] {#2} (#1,0);
\else
\draw (0,0) --  (#1,0);
\fi
\end{tikzpicture}
}

\bracketedpoints
\renewcommand{\questionlabel}{\textbf{\thequestion.}}
\renewcommand{\partlabel}{\textbf{\thepartno.}}
\renewcommand{\subpartlabel}{\textbf{\thesubpart.}}

\qformat{Question \thequestion\dotfill \emph{\totalpoints\ points}}

%\newcommand{\quest}[1][0]{\ifnum{#1}>0 \question[#1] \else \noaddpoints \question[\totalpoints] \addpoints \fi}

\vqword{Problem}
\vtword{Total}
\cellwidth{1in}

\pagestyle{head}

\firstpageheader{}{}{}
\runningheader{\scriptsize\textit{Math 5248 --- Midterm Exam 2 (Spring 2023)}}{}{\scriptsize\textit{page \thepage}}
\runningheadrule

\extrafootheight{-.5in}

%\printanswers

%%%%%%%%%%%%%%%%%%%%%%%%%%%%%%%%%%%%%%%%%%%%%%%%%%%

\begin{document}

\vspace*{-0.5in}

\ifprintanswers
\begin{center}
\fbox{\LARGE\textbf{SOLUTIONS}}
\end{center}
\fi

{\centering

\LARGE Math 5248 --- Midterm Exam 2

}

\smallskip

{\centering

\large Due by 11:59pm on April 21, 2022

}

\bigskip\bigskip


\hrule

\begin{enumerate}[leftmargin=2em,rightmargin=1em]\setlength{\itemsep}{-2pt}


\item This exam has a total of 11 required problems.  Additionally, there are 3 optional problems that will be graded for proficiency only. 


\item Your completed exam must be uploaded to Gradescope by 11:59 pm central standard time on April 21, 2023.  

\item You are allowed to use any non-human resources you can find, but you must cite any sources that you use. Your citation must give enough information for the instructor to find the source you used. 

\item All solutions must be written up using your own words. 

\item You may not use assistance from any humans besides Prof.~Burson. This includes posting the questions on sites such as Chegg or StackExchange.  
\item You must clearly explain your answers at a level that another MATH 5248 student could understand.  

\item You will not get credit for solutions that use major results that we have not covered yet in the course (email Prof.~Burson if you are unsure of what counts as a major result). 

\item If you have questions for Prof.~Burson, \href{https://calendar.google.com/calendar/selfsched?sstoken=UURMcDY3OWhoVDI3fGRlZmF1bHR8ZGMyODk0OWZhNmE4NzQ0ZmUwYmY0ODk2N2I3NDFiYzk}{you may sign up for an appointment here} or send her an email. 

\item If you would like to have some of your problems graded for proficiency please write a list of all problems you would like graded for proficiency credit and which skill(s) you used in each of those problems. \\
\emph{Proficiency credit information:}




\end{enumerate}



\vfill

(If you submit your exam on separate sheets of paper, please {\bf copy this statement in full} and sign. You only need to copy the statement once. ) 
\vspace{2em}\\
\emph{I neither gave nor received aide aside from the resources specifically allowed in item 3 during this exam.}

\vspace{1cm}

\hfill\textbf{Signature:} \answerblank{3in}{}

\vfill

\newpage
\begin{center}
\large
How the exam will be assessed
\end{center}
Each part of questions X-Y will be graded using one of the rubrics below. In the instructions for question Z, there is a summary of how your work will be graded.
\begin{center}
\begin{tabular}{|p{3.5cm}|p{2.7cm}|p{2.7cm}|p{2.5cm}|p{2.8cm}|}
\hline \multicolumn{5}{|c|}{4-point rubric}\\ \hline
4 & 3 & 2 &1 & 0\\ \hline
Correct and clearly communicated. Solution/proof demonstrates clear understanding of the concepts. If the problem requires a proof, the proof clearly states all assumptions, defines all variables, and notes where definitions or theorems are used. & Solution/proof demonstrates understanding of the concepts, but might have one minor issue such as a small computational error or a small jump in logic.  & Solution/proof demonstrates partial understanding, but has a large jump in logic or serious writing flaws such as incomplete sentences in a proof.  & Solution/proof shows effort but it has serious flaws in mathematical logic.  &Solution is not submitted, is illegible, or does not demonstrate evidence of understanding. \\ \hline
\end{tabular}
\begin{tabular}{|p{3cm}|p{3.9cm}|p{3cm}|p{2.8cm}|}
\hline \multicolumn{4}{|c|}{3-point rubric}\\ \hline
 3 & 2 &1 & 0\\ \hline
Solution demonstrates clear understanding of the concepts. Work and exposition allow the reader to easily follow the logic. & Solution demonstrates partial, but significant understanding of the concepts. It might have one or two minor issues such as computational errors or a small piece of missing justification. & Solution shows some understanding but it has serious flaws in mathematical logic.  &Solution is not submitted, is illegible, or does not demonstrate evidence of understanding. \\ \hline
\end{tabular}\\
\begin{tabular}{|p{3cm}|p{3.6cm}|p{3.4cm}|}
\hline \multicolumn{3}{|c|}{2-point rubric}\\ \hline
 2 &1 & 0\\ \hline
Solution demonstrates clear understanding of the concepts.  & Solution shows some understanding but it has serious flaws in mathematical logic.  &Solution is not submitted, is illegible, or does not demonstrate evidence of understanding. \\ \hline
\end{tabular}
\end{center}


\vspace{\stretch{1}}

\gradetablestretch{1.35}

{\centering

\partialpointtable{1} 
\\[2em]
You may find the following useful:
\begin{align*}
&A\hspace{.35cm} B\hspace{.35cm}C\hspace{.35cm}D\hspace{.35cm}E\hspace{.35cm}F\hspace{.35cm}G\hspace{.35cm}H\hspace{.35cm}I\hspace{.35cm}J\hspace{.35cm}K\hspace{.35cm}L\hspace{.35cm}M\hspace{.35cm}N\hspace{.35cm}O\hspace{.35cm}P\hspace{.35cm}Q\hspace{.35cm}R\hspace{.35cm}S\hspace{.35cm}T\hspace{.35cm}U\hspace{.35cm}V\hspace{.35cm}W\hspace{.35cm}X\hspace{.35cm}Y\hspace{.35cm}Z\\
&0\hspace{.46cm} 1\hspace{.46cm} 2\hspace{.46cm} 3\hspace{.46cm} 4\hspace{.46cm} 5\hspace{.46cm} 6\hspace{.46cm} 7\hspace{.46cm} 8\hspace{.46cm} 9\hspace{.35cm} 10\hspace{.285cm} 11\hspace{.285cm} 12\hspace{.285cm} 13\hspace{.285cm} 14\hspace{.285cm} 15\hspace{.285cm}16\hspace{.285cm}17\hspace{.285cm}18\hspace{.285cm}19\hspace{.285cm}20\hspace{.285cm}21\hspace{.285cm}22\hspace{.285cm}23\hspace{.285cm}24\hspace{.285cm}25
\end{align*}
}


The sage command for computing $a^b\%p$ is \verb|power_mod(a,b,p)|. The Mathematica/WolframAlpha command for the same computation is \verb|PowerMod(a,b,p)|. 
\vfill





\newpage

\begin{questions}
\begingradingrange{1}
\question[3] Determine whether or not $7$ is a primitive root of $31$ without computing all of the powers of $7$ modulo $31$.
\vfill

\question[3]Use the fact that $8^{4645}\equiv8\pmod{9291}$ to prove that $9291$ is not prime. 
\vfill 
\question[3]  Find the principal square root of $6$ modulo $11$, or explain why one does not exist. 
   
  \vfill
   \newpage
 \question Consider the following system of equations:
 $$\systeme*{8x+5y\equiv a \pmod{14}, bx+y\equiv 3 \pmod{14}},$$
where you will choose values for the constants $a$ and $b$ in each part of this problem. Make sure to clearly state your choice for $a$ and $b$ in each part.
\begin{parts}
\part[3] Choose integer values for $a$ and $b$ such that the system has one solution modulo $14$.  Then, find that solution to your system, making sure to clearly explain your steps. 
\vfill
\part[4] Choose integer values for $a$ and $b$ such that the system has multiple solutions modulo $14$.  Then, find the solutions to your system, making sure to clearly explain your steps.
\vfill 
\part[3] Choose integer values for $a$ and $b$ such that the system has no solutions. Explain why your system has no solutions. 
\end{parts}
\vfill
    \newpage



    \question You want to send messages using a Hill cipher with key $K=\begin{pmatrix}
6 & 3 \\7 &0
\end{pmatrix}.$ 

  \begin{parts}
\part[3]  
Encrypt the message ``hello all". 
\vfill
\part[4] Decrypt the ciphertext ``INXGXC."
\vfill
\part[4] An adversary intercepts your communication. However, you are lucky that the adversary was only able to figure out that the plaintext ``stop" corresponds to the ciphertext ``JWZU." They will find that there are two possible keys that could have been used.  What are those two keys?

  

   
 \end{parts}
  
   \vfill
   \newpage
\question Kate is creating an ElGamal cipher and chooses prime modulus $3917$, base $21$, and random exponent $68$.  
\begin{parts}
\part[2] What information does Kate publish for the public to see? (Make sure you include all information the sender needs from Kate to encrypt the message.)
\vfill
\part[3]You want to send Kate the message $789$ using her public key.  If you choose the exponent $k=26$ as your ephemeral key, what ciphertext do you send her?
\vfill
\part[3] Using Kate's public key, Dr. Burson encrypts a message and sends Kate the ciphertext $(306, 896)$. What was the integer value of the original message?
\vfill
\end{parts}
\vfill
\newpage

\question You want to recieve messages using the RSA algorithm with modulus $736313=991\cdot 743$. 
\begin{parts}
\part[2] Give an example of an exponent that would {\bf not} be a valid encryption exponent. Explain why your exponent will not work. 
\vfill
\part[3] Give an example of a valid encryption exponent and calculate the matching decryption exponent. 
\vfill
\end{parts}
\newpage
   
\question 
Dr. Burson publishes $(201967, 7)$ as her public key to receive messages using RSA encryption. 
\begin{parts}
\part[3] If you want to send Dr. Burson the message $m=842$, what ciphertext would you send?
\vspace{2in}
\part[4] You have an oracle that tells you that $432^2\equiv 201535^2 \pmod{201967}$. Can you use this information to factor $201967$? If so, do it. If not, explain why not. 

\end{parts}
\vfill
 \question[4] Prove that $x^{32}+51y=5$ has no integer solutions.
 \vfill
\newpage
 \question Let $p\ge 5$ be a prime. We say that $a$ is a \emph{cube modulo} $p$ if there is an integer $c$ satisfying $a\equiv c^3\pmod{p}$. 
 \begin{parts}
 \part[3] Let $a$ and $b$ be cubes modulo $p$. Prove that $ab$ is a cube modulo $p$. 
 \vfill
 \part[3] Give an example to show that (unlike the case with squares), it is possible for none of $a$, $b$, and $ab$ to be a cube modulo $p$. 
 \vfill
 \end{parts}
 \vfill
 \newpage
\question Suppose that $n$ is an integer possessing a primitive root and that $g$ is a primitive root modulo $n$.  Prove or disprove each of the following statements.
\begin{parts}
\part[4] If $k\equiv \ell\pmod{n}$, then $g^k\equiv g^\ell\pmod{n}$.
\vfill
\part[4] If $k\equiv \ell\pmod{\phi(n)}$, then $g^k\equiv g^\ell \pmod{n}$.
\vfill
\end{parts}
\newpage
The following questions are optional and will be graded for proficiency only. 
\question Prove that, if $g$ is a primitive root modulo $n$, then $g$ is a unit modulo $n$.
 \vfill
 \question Let $a$ and $b$ be positive integers. Prove that,  if $\gcd(a,b)=1$, then $\gcd(a^2,b)=1$. 
 \vfill
\question Fix a prime number $p$ and an integer $n\not\equiv 0,1\pmod{p}$. Set $x=n^{p-2}+n^{p-3}+\cdots+n+1$. Show that $p|x$. (Hint: consider expanding $(n-1)x$.)
\vfill
\endgradingrange{1}
\end{questions}


\end{document}