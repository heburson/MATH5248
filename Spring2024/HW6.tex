 \documentclass[12pt]{article}
\usepackage[utf8]{inputenc}
\usepackage{amsmath,amsthm}
\usepackage{fullpage}
\usepackage{amsfonts}
\usepackage{amssymb,multicol}
\usepackage[colorlinks=true,urlcolor=blue]{hyperref}
\usepackage{enumitem}
\usepackage{xcolor,systeme}

\newcommand{\Z}{\mathbb{Z}}
\newtheorem*{theorem}{Theorem}
\newtheorem*{lemma}{Lemma}

\newlist{checklist}{itemize}{2}
\setlist[checklist]{label=$\square$}

\begin{document}
\begin{center}
{\Large Homework 6}\\
Due: Friday,  March 22 at noon\\


\end{center}
{\bf Instructions:} Submit a pdf of your solutions to the HW 6 assignment on Gradescope.  {\bf You must complete the process by associating the proper page(s) to each question.} (If you fail to do so, you may earn ``not assessable" on any problems that do not have solutions matched to them.)\\[3pt]

{\bf Something new:} On this assignment, instead of writing what to focus on at the top, the main learning objectives being assessed and the grading criteria for each problem is included under the problem statement.  If you are unsure about the meaning of a criterion, ask Dr. Burson.  

\begin{enumerate}

\item Martha is sending a message talking about how great her sister is.  Martha has encrypted her message using a Hill cipher with a $2\times 2$ key.  You know that she has encrypted the word ``best" to the ciphertext ``JCTF".  What are all possible keys to Martha's Hill cipher? (There may be one or more possible keys.) To earn credit for this problem, you must use the Extended Euclidean Algorithm to find an inverse modulo 26 as an intermediate step. \\
{\bf Learning objectives:} (1) Execute a known plaintext attack on the Hill Cipher. (2) Find inverses modulo m using the Extended Euclidean Algorithm.  (3) Find inverses of matrices modulo m.\\
{\bf Grading Criteria:} Final answer is correct and the work leading up to it is clear, legible, and written for the audience of other MATH 5248 students; All work for the Extended Euclidean Algorithm is shown; No major computational, logical, factual, or semantic errors in the work.

\item Find all possible ordered pairs of integers $(x,y)$ in the standard transversal modulo 21 satisfying the following system of congruences:
$$\systeme*{9x+12y\equiv 3\pmod{21}, 11x+5y\equiv 12\pmod{21}}$$ or explain why no such integers exist.  (Hints: $21=3\cdot 7$; Even though it looks different, this problem is very similar to the activity from Day 13; Some students find it helpful to rewrite the system as a single congruence involving matrices; If you find a modular inverse when solving this problem, you may use any method you want, including a computer algebra system such as Mathematica/WolframAlpha or Sage.)\\
{\bf Learning objectives:} (1) Solve a system of multivariate congruences when there are zero or multiple solutions.  \\
{\bf Grading criteria:} Final answer is correct and the work leading up to it is clear, legible, and written for the audience of other MATH 5248 students; Any use of modular inverses is clearly shown or explained; No major computational, logical, factual, or semantic errors in the work.

\item Use Fermat's Little Theorem to prove that $9873$ is not a prime number.  (Note: You may use an online calculator such as WolframAlpha to compute large powers modulo $m$.)\\
{\bf Learning objectives:} (1) Use Fermat's Little Theorem to prove that an integer is not prime.  \\
{\bf Grading criteria:} Proof is clear, legible, and written following mathematical conventions (as explained in the writing guidelines); Proof clearly and correctly cites Fermat's Little Theorem; No major computational, logical, factual, or semantic errors in the work.


\item Using Fermat's Little Theorem, prove that the equation $r^{32}+17t=4$ has no solutions for integers $r$ and $t$. \\
{\bf Learning objectives:} (1) Use Fermat's Little Theorem clearly and correctly in a proof.  (2) Using properties of modular congruences or the division algorithm, prove that an equation has no integer solutions. \\
{\bf Grading criteria:} Proof is clear, legible, and written following mathematical conventions (as explained in the writing guidelines); Proof clearly and correctly cites Fermat's Little Theorem (don't forget to confirm hypotheses); Proof correctly uses either properties of modular congruences or the division algorithm, including proper notation.

\end{enumerate}

\end{document}