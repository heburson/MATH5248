   \documentclass[12pt]{article}
\usepackage[utf8]{inputenc}
\usepackage{amsmath,amsthm}
\usepackage{fullpage}
\usepackage{amsfonts}
\usepackage{amssymb,multicol}
\usepackage[colorlinks=true,urlcolor=blue]{hyperref}
\usepackage{enumitem}
\usepackage{xcolor}

\newcommand{\Z}{\mathbb{Z}}
\newtheorem*{theorem}{Theorem}

\newlist{checklist}{itemize}{2}
\setlist[checklist]{label=$\square$}

\begin{document}
\begin{center}
{\Large Homework 3}\\
Due: Friday,  February 16 at noon\\


\end{center}
{\bf Instructions:} Submit a pdf of your solutions to the HW 3 assignment on Gradescope.  {\bf You must complete the process by associating the proper page(s) to each question.} (If you fail to do so, you may earn ``not assessable" on all problems.)\\[3pt]

\noindent When working on this assignment, you should focus on the following goals:
\begin{itemize}
\item Demonstrate understanding of the definition of a unit modulo $m$.  
\item Demonstrate understanding of the Theorem about solutions to Linear Diophantine Equations.
\item Demonstrate understanding of the Euclidean Algorithm and how it can be used to find multiplicative inverses.
\item Demonstrate understanding of known plaintext attacks on the Affine cipher. 
\item Write clear and correct proofs that meet the Writing Guidelines posted on Canvas. Focus especially on the following guidelines/conventions:
\begin{checklist}
\item All conventions enumerated in the HW1 and 2 instructions.
\item Start sentences with English words, not symbols. (Tip: ``Observe that" and ``Note that" are really helpful phrases!)
\end{checklist}
\end{itemize}

\begin{enumerate}

\item Let $n$ be a positive integer. Prove that $n$ is a unit modulo $5n-1$. 
\item Does the Linear Diophantine Equation $52x+36y=537$ have any solutions? Prove your answer.

\item Use the Euclidean Algorithm to find a multiplicative inverse of $206$ modulo $5427$. 

\item Two enemies of yours are passing messages using an affine cipher. You know that they always write formal notes starting with the greeting ``Hello" in the plaintext.  You intercept a ciphertext that starts with ``fkhhc." What key did they use?


\end{enumerate}

\end{document}
