   \documentclass[12pt]{article}
\usepackage[utf8]{inputenc}
\usepackage{amsmath,amsthm}
\usepackage{fullpage}
\usepackage{amsfonts}
\usepackage{amssymb,multicol}
\usepackage[colorlinks=true,urlcolor=blue]{hyperref}
\usepackage{enumitem}
\usepackage{xcolor}

\newcommand{\Z}{\mathbb{Z}}
\newtheorem*{theorem}{Theorem}

\newlist{checklist}{itemize}{2}
\setlist[checklist]{label=$\square$}

\begin{document}
\begin{center}
{\Large Homework 2}\\
Due: Friday,  February 10 at noon\\


\end{center}
{\bf Instructions:} Submit a pdf of your solutions to the HW 2 assignment on Gradescope. 

When working on this assignment, you should focus on the following goals:
\begin{itemize}
\item Demonstrate that you understand how to write a complete set of equivalence classes modulo $m$ and when two equivalence classes are equal. 
\item Demonstrate understanding of the definition of a unit modulo $m$.  
\item Use examples to determine if a conjecture is true and then prove or disprove the conjecture. 
\item Write clear and correct proofs that meet the Writing Guidelines posted on Canvas. Focus especially on the following guidelines/conventions:
\begin{checklist}
\item Written in complete sentences.
\item All assumptions stated at the beginning of the proof.
\item Define variables before/when you use them.
\item Use symbols when writing a precise mathematical formula or equation and English words when not writing a formula (i.e. do not use a symbol, such as $\forall,\, \exists,\, \therefore, \, \implies$ in place of a word or phrase.)
\item Show computational/algebraic work vertically and centered in the page. 
\end{checklist}
\end{itemize}

\begin{enumerate}

\item Suppose you know that the plaintext ``snow" has been encoded to the ciphertext ``ILXM." Explain why the cipher used could not have been a shift cipher.
\item Consider an affine cipher with key $(5,4)$. 
\begin{enumerate}
\item Encrypt the word ``cryptology" using that cipher. 
\item You receive the ciphertext ``DAROVSWR". What linear congruence equations must you solve to decrypt this text?
\end{enumerate}


\item Prove or disprove each of the following statements.
\begin{enumerate}
\item For integers $a$, $b$, $c$, and $m$ with $m,c>0$, if $a\equiv b\pmod{mc}$, then $a\equiv b\pmod{m}$. 
\item For integers $a$, $b$, $c$, and $m$ with $m,c>0$, if $a\equiv b\pmod{m}$, then $a\equiv b\pmod{mc}$. 
\end{enumerate} 
\item Suppose that $a=bq+r$, for some integers $a,b, q$ and $r$. Without using any properties of the $\gcd$ besides the definition as the largest common divisor of $x$ and $y$, prove that $\gcd(a,b)=\gcd(b,r)$.  (Hint: Prove that the set of common divisors of $a$ and $b$ is exactly the same as the set of common divisors of $b$ and $r$.)

\end{enumerate}

\end{document}
