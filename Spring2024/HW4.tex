   \documentclass[12pt]{article}
\usepackage[utf8]{inputenc}
\usepackage{amsmath,amsthm}
\usepackage{fullpage}
\usepackage{amsfonts}
\usepackage{amssymb,multicol}
\usepackage[colorlinks=true,urlcolor=blue]{hyperref}
\usepackage{enumitem}
\usepackage{xcolor}

\newcommand{\Z}{\mathbb{Z}}
\newtheorem*{theorem}{Theorem}

\newlist{checklist}{itemize}{2}
\setlist[checklist]{label=$\square$}

\begin{document}
\begin{center}
{\Large Homework 4}\\
Due: Friday,  February 23 at noon\\


\end{center}
{\bf Instructions:} Submit a pdf of your solutions to the HW 3 assignment on Gradescope.  {\bf You must complete the process by associating the proper page(s) to each question.} (If you fail to do so, you may earn ``not assessable" on all problems.)\\[3pt]

\noindent When working on this assignment, you should focus on the following goals:
\begin{itemize}
\item Demonstrate understanding of Euclid's Lemma.
\item Demonstrate understanding of how to compute the index of coincidence.
\item Demonstrate understanding of how to use the Mutual Index of Coincidence to determine information about a Vigen\'ere Cipher Key.
\item Write clear and correct proofs and disproofs (explanations of counterexamples) that meet the Writing Guidelines posted on Canvas. Focus especially on the following guidelines/conventions:
\begin{checklist}
\item All conventions enumerated in the HWs1-3 instructions.
\item Write full and properly formed sentences, and include all equations in a sentence (including proper punctuation at the end).
\end{checklist}
\end{itemize}

\begin{enumerate}
\item This exercise will help us work towards understanding square roots, which will come up later in the semester. 
\begin{enumerate}
\item Let $p$ be a prime number and let $b$ be an integer such that $x^2\equiv b\pmod{p}$ has a solution.  Show that,  if $a_1$ and $a_2$ are two such solutions, then $a_1\equiv a_2\pmod{p}$ or $a_1\equiv -a_2\pmod{p}$. 
\item Provide a counterexample to show that the above statement is not necessarily true if $p$ is not prime. In other words, provide integers $m$, $b$, $a_1$, and $a_2$ such that $a_1^2\equiv b\pmod{m}$, $a_2^2\equiv b\pmod{m}$, but $a_1\not \equiv \pm a_2 \pmod{m}$. Make sure you explain why your counterexample is, in fact, a counterexample. 
\end{enumerate}
\item Consider an alphabet with four characters: $\spadesuit$, $\heartsuit$, $\diamondsuit$, and $\clubsuit$.  Suppose that you have a text $\underline{s}$ written in this alphabet which consists of $60$ $\spadesuit$s,  $14$ $\heartsuit$s, $20$ $\diamondsuit$s, and $6$ $\clubsuit$s.  What is $\textrm{IndCo}(\underline{s})$?
\item  (You started this problem in class) You are trying to attack a Vigenere cipher.  You have already determined that this Vigenere cipher has key length $4$.  If $\underline{s} = (s_1, \ldots, s_{N})$ is a string of ciphertext of length $N$ with $N \equiv 0 \pmod4$, let $s^{(j)} = (s_j,s_{4+j},s_{8+j},\ldots,s_{N-4+j})$ be the substring of $s$ whose typical entry is $s_i$ for some $i \equiv j \pmod 4$.  Let $E_t(s^{(j)})$ be the encryption of the string $s^{(j)}$ by the shift cipher with key $t$ for $0 \leq t \leq 25$ and $1 \leq j \leq 4$ (i.e. the string obtained by taking $s^{(j)}$ and shifting each character forward by $t$ letters).  You compute the following table:

\begin{center}
    \begin{tabular}{ l | l | l | l }
    $t$ & $\mathrm{MutIndCo}(s^{(1)},E_t(s^{{(2)}}))$ & $\mathrm{MutIndCo}(s^{(1)},E_t(s^{{(3)}}))$  & $\mathrm{MutIndCo}(s^{(1)},E_t(s^{{(4)}}))$  \\ \hline
    3 & 0.057 & 0.031 & 0.036 \\     
    5 & 0.035 & 0.041 & 0.054 \\ 
    8& 0.052 & 0.072 & 0.042\\ 
    11 & 0.064 & 0.058 & 0.039\\ 
    16 & 0.029 & 0.065 & 0.068 \\ 
    23 & 0.035 & 0.037 & 0.051
    \end{tabular}
      \begin{tabular}{ l | l | l}
    $t$ & $\mathrm{MutIndCo}(s^{(2)},E_t(s^{{(3)}}))$ & $\mathrm{MutIndCo}(s^{(2)},E_t(s^{{(4)}}))$\\ \hline
    3 & 0.033 & 0.041 \\     
    5 & 0.039 & 0.069\\ 
    8 & 0.051 & 0. 038\\ 
    11 & 0.037 & 0.061\\ 
    16 & 0.045 & 0.052\\ 
    23 & 0.061 & 0.035\\
    \end{tabular}
    \end{center} 
    
For the questions below, assume that,  values of $t$ not appearing in the table above gave rise values of $\mathrm{MutIndCo}(s^{(j)},E_t(s^{(k)}))$ that are unremarkable for all $1 \leq j<k \leq 4$.  
\begin{enumerate}
\item Given that the key has the form $k = (k_1, k_2 k_3,k_4)$, based on the data above, what are all likely values of $k_2-k_1$, $k_3-k_1$, and $k_4-k_1$? Explain your answer.
\item  For what value(s) of $t$, would you expect $\mathrm{MutIndCo}(s^{(3)},E_t(s^{(4)}))$ to be large? Explain your answer.
\end{enumerate}
 \item Count the number of simple substitution ciphers on the English alphabet which map \textit{exactly} two letters to themselves. Show your work. (Note: This is just a minor variation of the problem done in class. Use that framework to help you and get the ball rolling.)

\end{enumerate}

\end{document}