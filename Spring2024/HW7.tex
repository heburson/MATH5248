 \documentclass[12pt]{article}
\usepackage[utf8]{inputenc}
\usepackage{amsmath,amsthm}
\usepackage{fullpage}
\usepackage{amsfonts}
\usepackage{amssymb,multicol}
\usepackage[colorlinks=true,urlcolor=blue]{hyperref}
\usepackage{enumitem}
\usepackage{xcolor,systeme}

\newcommand{\Z}{\mathbb{Z}}
\newtheorem*{theorem}{Theorem}
\newtheorem*{lemma}{Lemma}

\newlist{checklist}{itemize}{2}
\setlist[checklist]{label=$\square$}

\begin{document}
\begin{center}
{\Large Homework 7}\\
Due: Friday,  April 5 at noon\\


\end{center}
{\bf Instructions:} Submit a pdf of your solutions to the HW 7 assignment on Gradescope.  {\bf You must complete the process by associating the proper page(s) to each question.} (If you fail to do so, you may earn ``not assessable" on any problems that do not have solutions matched to them.)\\[3pt]

On this assignment, the main learning objectives being assessed and the grading criteria for each problem is included under the problem statement.  If you are unsure about the meaning of a criterion, ask Dr. Burson.  

\begin{enumerate}


\item Determine whether or not $3$ is a primitive root modulo $19$ without computing all of the powers of $3$ modulo $19$. \\
{\bf Learning objectives:} (1) Use the theorem connecting orders and primitive roots modulo $m$ to determine if an integer is a primitive root modulo another integer. (2) Use the theorem about possible orders modulo $m$ to compute the order of an element with as few computations as possible.\\
{\bf Grading Criteria:} Final answer is correct and the work leading up to it is clear, legible, and written for the audience of other MATH 5248 students; Any theorems used are clearly identified (it should be clear where you used the theorems mentioned in the learning objectives); No major computational, logical, factual, or semantic errors in the work.

\item Explain why $6$ is not a primitive root modulo $34$ without computing \emph{any} of the powers of $6$ modulo $34$. \\
{\bf Learning objectives:} (1) Demonstrate understanding of the requirements for being a primitive root modulo $m$.\\
{\bf Grading Criteria:} Final answer is correct and the work leading up to it is clear, legible, and written for the audience of other MATH 5248 students; A connection to the concept of units is made in the explanation; No major computational, logical, factual, or semantic errors in the work.  

\item Prove that, if $a$ is a primitive root modulo $p$, then $a^{-1}$ (the multiplicative inverse of $a$ modulo $p$) is also a primitive root modulo $p$. \\
{\bf Learning objectives:} (1) Use the theorem connecting primitive roots to orders in a proof.  (2) Write a clear and correct proof by contradiction. \\
{\bf Grading criteria:} Proof is clear, legible, and written following mathematical conventions (as explained in the writing guidelines); Proof clearly and correctly cites any theorems and definitions used (it should be clear if a claim is due to a theorem or a definition); Proof does not use any theorems that have not been covered in MATH 5248; Proof correctly uses either properties of modular congruences or the division algorithm, including proper notation.


\item Bob and Alice would like to use the Diffie-Hellman Key Exchange to agree on a session key $k$. Alice and Bob agree on the prime $6829$ and that the base $2$ modulo $6829$. Bob chooses a secret random number $x$ and tells Alice that $2^x\%6829$ is $5792$. Alice chooses $3$ as her secret exponent.  Answer the following questions without computing the value of $x$. 
\begin{enumerate}
\item What is the shared secret key $k$ that Alice and Bob will be using?
\item You are able to gain access to the network Alice and Bob are using to communicate, so you decide to implement a interceptor attack with exponent $c=10$. What key does Bob think he and Alice agreed to? What key does Alice think they agreed to?
\end{enumerate}	
{\bf Learning objectives:} (1) Implement the method of Diffie-Hellman Key Exchange. (2) Implement an interceptor attack on Diffie-Hellman Key Exchange.\\
{\bf Grading criteria:} Final answer is correct and the work leading up to it is clear, legible, and written for the audience of other MATH 5248 students; Work for part (a) does not require one to know the value of $x$; No major computational, logical, factual, or semantic errors in the work.  

\end{enumerate}

\end{document}