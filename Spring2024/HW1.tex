   \documentclass[12pt]{article}
\usepackage[utf8]{inputenc}
\usepackage{amsmath,amsthm}
\usepackage{fullpage}
\usepackage{amsfonts}
\usepackage{amssymb,multicol}
\usepackage{hyperref, enumitem}
\usepackage{xcolor,graphicx}

\newcommand{\Z}{\mathbb{Z}}
\newtheorem*{theorem}{Theorem}

\newlist{checklist}{itemize}{2}
\setlist[checklist]{label=$\square$}

\begin{document}
\begin{center}
{\Large Homework 1}\\
Due: Friday,  February 2 at noon CST\\


\end{center}
{\bf Instructions:} Submit a pdf of your solutions to the HW 1 assignment on Gradescope. 

When working on this assignment, you should focus on the following goals:
\begin{itemize}
\item Demonstrate that you understand the definition of reduction modulo $m$ and how it relates to the division algorithm. 
\item Clearly communicate computational solutions using complete sentences and enough explanation that another 5248 student could follow your work.
\item Clearly and correctly apply the definition of divisibility in a proof. 
\item Clearly and correctly apply the division algorithm in a proof.
\item Write clear and correct proofs that meet the following conventions of a mathematical proof:
\begin{checklist}
\item Written in complete sentences.
\item All assumptions stated at the beginning of the proof.
\item Define variables before/when you use them.
\item Use symbols when writing a precise mathematical formula or equation and English words when not writing a formula (i.e. do not use a symbol, such as $\forall,\, \exists,\, \therefore, \, \implies$ in place of a word or phrase.)
\item Show computational/algebraic work vertically and centered in the page. 
\end{checklist}
\end{itemize}

\begin{enumerate}
\item Compute the following reductions.  Explain your steps.  For parts (a) and (c), show your work without using any integers larger than 144. 
\begin{enumerate}

\item $(131\cdot 142) \% 3$ \\
\item $(-2000) \% 93$\\
\item $\left(3^{74}\right) \% 13$ 
\end{enumerate}
\item
\begin{enumerate}
\item  Let $N$ be an integer so that $N\%24 = 2$.  What is $N\%6$?  Prove that your answer is true for any integer $N$ such that $N\%24=2$.
\item Fix an integer $N$ such that $N\%6=2$. What are the possible values of $N\%24$? Explain your answer. 
\end{enumerate}

\item Using the division algorithm, show that $x^3+2=0$ cannot have any integer solutions. (Note: you should have most of this problem completed from the in-class activity.)

\item Let $a$, $b$, and $c$ be integers such that $a|b$ and $a|c$. Prove that,  for any integers $u$ and $v$, $a|ub+vc$. 
\item Let $m$ be any positive integer. Prove that  if $r$ is the reduction of $N$ modulo $m$ with $r\ne 0$, then $m-r$ is the reduction of $-N$ modulo $m$.  (Note: you likely have a lot of this problem completed from the in-class activity!)


\end{enumerate}

\end{document}
