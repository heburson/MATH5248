   \documentclass[10pt,a4paper]{article}
\usepackage[utf8]{inputenc}
\usepackage{amsmath}
\usepackage{fullpage}
\usepackage{amsfonts}
\usepackage{amssymb,systeme}
\usepackage[colorlinks = true,
            linkcolor = blue,
            urlcolor  = blue,
            citecolor = blue,
            anchorcolor = blue]{hyperref}
\usepackage{xcolor}

\newcommand{\Z}{\mathbb{Z}}
\newcommand{\ds}{\displaystyle}
\begin{document}
\begin{center}
{\Large Homework 5: Computations}\\
Due: Friday October 29, 2021 at Noon\\


\end{center}
{\bf Instructions:} Submit a pdf of your solutions to the HW 5 Computational assignment on Canvas. Remember that your work will be graded based on how well it meets \href{https://docs.google.com/document/d/1emM06_WRh_h941rsjtRE9fRVndJtfRKd9gyS3Fs_rFA/edit?usp=sharing}{the specifications. }


{\bf Note:} It is possible that you will need to explain why no solutions exist to a system of equations. This is the sort of problem that straddles computational vs proof/theory. It is important that your explanation is written in complete sentences and explains why the system of equations cannot have a solution. 

\begin{enumerate}

\item Solve each of the following simultaneous systems of congruences or explain
why no solution exists.
\begin{enumerate}
\item $\systeme*{x\equiv 5\pmod{13}, x\equiv 2\pmod{7}\;\;, x\equiv 4 \pmod{11}}$
\item $\systeme*{x\equiv 3\pmod{9}, x\equiv 2\pmod 6, x\equiv 1\pmod{5}}$
\end{enumerate}
   
 \item  A friend sends you the encrypted message $XI$ using a Hill cipher whose key you and your friend agreed at a previous meeting would be $K = \begin{pmatrix}
1 & 2\\ 
4 & 17
\end{pmatrix}$.  What is the plaintext of the message your friend sent you?

\item For each of the following systems of congruences, either find integers $x,y$ satisfying the system or explain why no such integers exist. 
\begin{enumerate}
\item  $\ds
\systeme*{9x+3y \equiv 6 \pmod {13}, x+8y \equiv 2 \pmod {13}}$
\item $\ds \systeme*{19x+100y \equiv 74 \pmod {323}, 228x+83y \equiv 38 \pmod {323}}$
\end{enumerate}
\end{enumerate}
\end{document}