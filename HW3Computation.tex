   \documentclass[10pt,a4paper]{article}
\usepackage[utf8]{inputenc}
\usepackage{amsmath}
\usepackage{fullpage}
\usepackage{amsfonts}
\usepackage{amssymb}
\usepackage[colorlinks = true,
            linkcolor = blue,
            urlcolor  = blue,
            citecolor = blue,
            anchorcolor = blue]{hyperref}
\usepackage{xcolor}

\newcommand{\Z}{\mathbb{Z}}
\begin{document}
\begin{center}
{\Large Homework 3: Computations}\\
Due: Friday October 8, 2021 at Noon\\


\end{center}
{\bf Instructions:} Submit a pdf of your solutions to the HW 3 Computational assignment on Canvas. Remember that your work will be graded based on how well it meets \href{https://docs.google.com/document/d/1emM06_WRh_h941rsjtRE9fRVndJtfRKd9gyS3Fs_rFA/edit?usp=sharing}{the specifications. }
\\[1em]

\noindent {\bf Further instructions:} Problems 1 and 2 require you to find multiplicative inverses modulo 26. To earn a passing grade, you must show all your steps to find the inverse using the Euclidean Algorithm for at least one of the problems. 
For nonzero , prove that if  and  are coprime, and likewise  and  are coprime, then  and  are coprime. 
\begin{enumerate}
\item Consider an affine cipher with key $(5,4)$. 
\begin{enumerate}
\item Encrypt the word ``cryptology" using that cipher. 
\item You recieve the ciphertext ``DAROVSWR". What is the decrypted plaintext?
\end{enumerate}

\item Two enemies of yours are passing messages using an affine cipher. You know that they always write formal notes starting with the greating ``Hello" in the plaintext.  You intercept a ciphertext that starts with ``fkhhc." What key did they use?

\item By hand, find the prime factorizations of 756 and 1001. Then use those factorizations to find $\gcd(756, 1001)$ and $\mathrm{lcm}(756, 1001)$. (You may use the formulas we gave in class without proof.) 

\end{enumerate}
\end{document}